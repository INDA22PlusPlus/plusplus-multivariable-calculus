\documentclass{report}

\include{preamble}
\include{macros}
\include{letterfonts}

\usepackage[Glenn]{fncychap}

\title{\Huge{SF1626 Flervariabelanalys, bra formler}}
\author{\huge{}}
\date{}


\begin{document}

\maketitle
\newpage
\pagebreak

\chapter{Satser, definitioner}
\ex{När Jacobianen inte behövs!}
{
	Om man behöver beräkna ytintegralen eller även flödesintegralen av en kurva så behöver man inte lägga till Jacobianen!\\\\
	
	\textbf{Till exempel}: Om man vill beräkna ytan av kurvan som beskrivs som intersektionen av cylindern $ x^2+y^2 \le 1$ och ytan $x^2+z^2 = 1, \: z \ge 0$, så kan parametrisera kurvan på sättet nedan:
	\begin{equation*}
	\vec{r}(t) =
	\begin{cases}
		x &= rcos(t)\\
		y &= rsin(t)\\
		z &= \sqrt{1-r^2cos^2(t)}
	\end{cases}
	\end{equation*}
	...$r \in [0,1], \:\: t \in [0, 2\pi]$. Om man då behöver beräkna ytan så gör man det på sättet nedan:
	\begin{equation*}
		\iint_Y \Bigl| \frac{\partial \vec{r}}{\partial r} \times \frac{\partial \vec{r}}{\partial t} \Bigr| \: drdt = \int_0^{2\pi}\int_0^1 \Bigl| \frac{\partial \vec{r}}{\partial r} \times \frac{\partial \vec{r}}{\partial t} \Bigr| \: drdt
	\end{equation*}
	Lägg märke på att Jacobianen inte lades till! Detta på grund att vi \textbf{inte} transformerar en areaelement från en koordinatsystem till en annan - eftersom vi från första början hade parametriserat kurvan i koordinatsystemet vi behöver för att beräkna ytan. Mer info: \href{https://math.stackexchange.com/questions/2388245/why-is-there-no-jacobian-in-the-definition-of-the-surface-integral-iint-ufds}{[1]} \href{https://math.stackexchange.com/a/2332951}{[2]}. 
}

\thm{Gradientens definition}
{
	\begin{equation*}
		\nabla F( \vec{a} ) = \lim_{t \to 0} \frac{ F(\vec{a} + t(e_1, e_2, e_3, \cdots, e_n) - F(\vec{a})) }{t}
	\end{equation*}
}

\dfn{Tangentplan (linjär approximation) unmystified}
{
	I vissa uppgifter så kommer det att frågas om att bestämma en tangentplan i en viss punkt för en kurva som ges implicit, t.ex: $ cos(2z)(x^2+3y^2+z) = 0 $, detta kan tolkas som en 3D kurva. Då tangentplanen i en punkt $\vec{a} \in \mathbb{R}^n $, för en kurva $f = k$, $f : \mathbb{R}^n \mapsto \mathbb{R} $ ges av:
	\begin{equation*}
		\nabla 	f(\vec{a}) \cdot (\vec{x} - \vec{a})
	\end{equation*}
	..där $\vec{a} = (a_1, a_2, a_3, \cdots, a_n)$ och $\vec{x} = (x_1, x_2, x_3, \cdots, x_n)$. I $\mathbb{R}^3$ så blir $\vec{x} = (x,y,z) $.\\\\
	
	\textbf{Observera} att funktioner som $g(x,y) = z$ kan skrivas om implicit till $g(x,y) - z = 0$ för att använda denna metod.\\
	Mer info: \href{https://math.stackexchange.com/a/2084635}{[1]}
}

\thm{Orienteringsidentitet på \textbf{vektorfälts}integral}
{
	\begin{equation*}
		\ointctrclockwise_{\gamma} F \cdot d\vec{r} = -\varointctrclockwise_{\gamma} F \cdot d\vec{r} \iff \varointctrclockwise_{\gamma} F \cdot d\vec{r} = -\ointctrclockwise_{\gamma} F \cdot d\vec{r}
	\end{equation*}
	Mer info: \href{https://en.wikipedia.org/wiki/Vector_calculus_identities}{[1]}
}


\end{document}
