\documentclass{report}

\include{preamble}
\include{macros}
\include{letterfonts}

\usepackage[Glenn]{fncychap}

\title{\Huge{SF1626 Flervariabelanalys, bra formler}}
\author{\huge{}}
\date{}


\begin{document}

\maketitle
\newpage
\pagebreak

\chapter{Satser, definitioner}
\ex{När Jacobianen inte behövs!}
{
	Om man behöver beräkna ytintegralen eller även flödesintegralen av en kurva så behöver man inte lägga till Jacobianen!\\\\
	
	\textbf{Till exempel}: Om man vill beräkna ytan av kurvan som beskrivs som intersektionen av cylindern $ x^2+y^2 \le 1$ och ytan $x^2+z^2 = 1, \: z \ge 0$, så kan parametrisera kurvan på sättet nedan:
	\begin{equation*}
	\vec{r}(t) =
	\begin{cases}
		x &= rcos(t)\\
		y &= rsin(t)\\
		z &= \sqrt{1-r^2cos^2(t)}
	\end{cases}
	\end{equation*}
	...$r \in [0,1], \:\: t \in [0, 2\pi]$. Om man då behöver beräkna ytan så gör man det på sättet nedan:
	\begin{equation*}
		\iint_Y \Bigl| \frac{\partial \vec{r}}{\partial r} \times \frac{\partial \vec{r}}{\partial t} \Bigr| \: drdt = \int_0^{2\pi}\int_0^1 \Bigl| \frac{\partial \vec{r}}{\partial r} \times \frac{\partial \vec{r}}{\partial t} \Bigr| \: drdt
	\end{equation*}
	Lägg märke på att Jacobianen inte lades till! Detta på grund att vi \textbf{inte} transformerar en areaelement från en koordinatsystem till en annan - eftersom vi från första början hade parametriserat kurvan i koordinatsystemet vi behöver för att beräkna ytan. Mer info: \href{https://math.stackexchange.com/questions/2388245/why-is-there-no-jacobian-in-the-definition-of-the-surface-integral-iint-ufds}{[1]} \href{https://math.stackexchange.com/a/2332951}{[2]}. 
}

\thm{Gradientens definition}
{
	\begin{equation*}
		\nabla F( \vec{a} ) = \lim_{t \to 0} \frac{ F(\vec{a} + t(e_1, e_2, e_3, \cdots, e_n) - F(\vec{a})) }{t}
	\end{equation*}
}

\dfn{Tangentplan (linjär approximation) unmystified}
{
	I vissa uppgifter så kommer det att frågas om att bestämma en tangentplan i en viss punkt för en kurva som ges implicit, t.ex: $ cos(2z)(x^2+3y^2+z) = 0 $, detta kan tolkas som en 3D kurva. Då tangentplanen i en punkt $\vec{a} \in \mathbb{R}^n $, för en kurva $f = k$, $f : \mathbb{R}^n \mapsto \mathbb{R} $ ges av:
	\begin{equation*}
		\nabla 	f(\vec{a}) \cdot (\vec{x} - \vec{a})
	\end{equation*}
	..där $\vec{a} = (a_1, a_2, a_3, \cdots, a_n)$ och $\vec{x} = (x_1, x_2, x_3, \cdots, x_n)$. I $\mathbb{R}^3$ så blir $\vec{x} = (x,y,z) $.\\\\
	
	\textbf{Observera} att funktioner som $g(x,y) = z$ kan skrivas om implicit till $g(x,y) - z = 0$ för att använda denna metod.\\
	Mer info: \href{https://math.stackexchange.com/a/2084635}{[1]}
}

\thm{Orienteringsidentitet på \textbf{vektorfälts}integral}
{
	\begin{equation*}
		\ointctrclockwise_{\gamma} F \cdot d\vec{r} = -\ointclockwise_{\gamma} F \cdot d\vec{r} \iff \ointclockwise_{\gamma} F \cdot d\vec{r} = -\ointctrclockwise_{\gamma} F \cdot d\vec{r}
	\end{equation*}
	Mer info: \href{https://en.wikipedia.org/wiki/Vector_calculus_identities}{[1]}
}

\dfn{Orientering på randkurvor och randytor}
{
	En rand\textbf{yta} sägs vara \textbf{positiv} orienterad om randytans normal pekar ifrån själva kroppen som randytan täcker. Tvärtom med negativ orientering på randyta.\\\\
	
	En rand\textbf{kurva} sägs vara \textbf{positiv} orienterad om ytan som randkurvan täcker är åt vänstra sidan när man följer orienteringen på kurvan. Tvärtom med negativ orientering på randkurvan.
}


\thm{Inversa funktionssatsen}
{
	Om vi har en funktion $f : \mathbb{R}^n \mapsto \mathbb{R}^n$ och vi betecknar Jacobianen av $f$ i punkten $\vec{a}$ som $Df(\vec{a})$, så är $f$ inverterbar i närheten av punkten $\vec{a}$ om $Df(\vec{a})$ är inverterbar, d.v.s:
	\begin{equation*}
		det(Df(\vec{a})) \ne 0
	\end{equation*}
	Dessutom så gäller identiteten:
	\begin{equation*}
		D(f^{-1})(\vec{y}) = [Df(\vec{x})]^{-1}
	\end{equation*}
	..där $ f(\vec{x}) = \vec{y} $\\\\
	
	Mer info: \href{http://www.math.toronto.edu/courses/mat237y1/20189/notes/Chapter3/S3.3.html}{[1]}
}

\dfn{Linjäriseing för en funktion $f : \mathbb{R}^n \mapsto \mathbb{R}$}
{
Linjäriseringen av en funktion $f$ i en punkt $\vec{a}$ kan beskrivas på sättet nedan:
\begin{equation*}
	L(\vec{x})_{\vec{a}} = f(\vec{a}) + \nabla f_{\vec{a}} \cdot (\vec{x}-\vec{a})
\end{equation*}
}

\dfn{Differentierbarhet för funktioner $f : \mathbb{R}^n \mapsto \mathbb{R}$}
{
Observera att differentierbarhet inte är samma sak som deriverbar! Differentierbarheten för funktionen $f$ i en punkt $\vec{a}$ definieras som:
\begin{equation*}
\lim_{\vec{h} \to \vec{0}} \frac{f(\vec{a}+\vec{h})-L(\vec{a}+\vec{h})}{||h||} = 0
\end{equation*}
..alltså att linjäriseringen för funktionen går mot det riktiga värdet av funktionen. 
}

\dfn{Greens sats med alla villkor}
{
	Låt $D$ vara en en sluten och begränsad (kompakt) mängd i $\mathbb{R}^2$, där randkurvan $\gamma$ till $D$ består av en eller flera stängda styckvis kontinuerligt deriverbara kurvor (\textit{en.} piecewise smooth). Alltså att randkurvorna är deriverbara överallt förutom i vissa punkter, som kan tolkas som hörn till kurvorna. Randkurvorna ska då vara positivt orienterade med avseende på $D$. Låt vektorfältet $F$ vara $C^1$ i hela $D$, då gäller följande:
	\begin{equation*}
		\oint_{\gamma} F_1(x,y)dx + F_2(x,y)dy = \iint_D \bigl(\frac{\partial F_2}{\partial x} - \frac{\partial F_1}{\partial y}\bigr) dA
	\end{equation*}
}	

\dfn{Divergenssatsen (Gauss sats) med alla villkor}
{
	Låt $D$ vara en sluten och begränsad (kompakt) mängd i $\mathbb{R}^3$, där randytan $S$ är stängd och positivt orienterad (normalen pekar ut från $D$), samt styckvis glatt med enhetsnormalfältet $\hat{N}$. Låt också vektorfältet $F$ vara $C^1$ i hela $D$, då gäller följande:
	\begin{equation*}
		\iiint_D \nabla \cdot F dV = \oiint_S F \cdot \hat{N} dS
	\end{equation*}
}

\dfn{Stokes sats med alla villkor}
{
	Låt $D$ vara en styckvis glatt yta i $\mathbb{R}^3$ med randkurvan $\gamma$. Låt $D$ ha enhetsnormalfältet $\hat{N}$. Låt randkurvan $\gamma$ bestå av en eller flera stängda styckvis kontinuerligt deriverbara kurvor som är positivt orienterade med hänsyn till $D$. Låt vektorfältet $F$ vara $C^1$ och definierad på $D$, då gäller följande:
	\begin{equation*}
		\oint_{\gamma} F \cdot d\vec{r} = \iint_D (\nabla \times F) \cdot \hat{N} dS
	\end{equation*}
}

\pagebreak

\chapter{Bevis, bevistekniker}
\bt{Bevis på att $D_{\vec{u}}f(\vec{a}) = \vec{u} \cdot \nabla f(\vec{a})$, (15/03/2022, fråga 6)}
{
	Låt $f : \mathbb{R}^n \mapsto \mathbb{R}$, $\vec{a} = (a_1, a_2, \cdots, a_n)$ och $\vec{u} = (u_1, u_2, \cdots, u_n)$. Då kan vi definiera $D_{\vec{u}}f(\vec{a})$ med ett gränsvärde på följande sätt:
	\begin{equation*}
		\lim_{t \to 0} \frac{ f(\vec{a} + t\vec{u})-f(\vec{a})}{t}
	\end{equation*}
	Låt oss definiera $g(t) = f(\vec{a} + t\vec{u})$, då kan vi skriva gränsvärdet ovan som:
	\begin{equation*}
		\lim_{t \to 0} \frac{g(t)-g(0)}{t} = \frac{dg}{dt}(0)
	\end{equation*}
	Det sista kan utvecklas vidare till:
	\begin{align*}
		\frac{dg}{dt} &= \frac{d}{dt}f(\vec{a} + t\vec{u})\\
			      &= u_1 \frac{\partial f}{x_1} + u_2 \frac{\partial f}{x_2} + \cdots + u_n \frac{\partial f}{x_n}\\
			      &= \vec{u} \cdot \bigl( \frac{\partial f}{x_1}, \frac{\partial f}{x_2}, \cdots, \frac{\partial f}{x_n} \bigr)\\
			      &= \vec{u} \cdot \nabla f(\vec{a} + t\vec{u})
	\end{align*}
	..när man sätter $ t = 0$ i det sista så får vi $\vec{u} \cdot \nabla f(\vec{a})$. Alltså \textbf{V.S.B}!
}

\bt{Gradienten och nabla operatorn i polära koordinater (genom tillämpningen av kedjeregeln)}
{
	För att beräkna gradienten, så behöver vi en ortogonal koordinatsystem. I polära koordinater, brukar man definiera enhets-basvektorerna som beror beror på vinkeln $\theta$ som:
	\begin{align*}
		\boldsymbol{\hat{r}} &= \cos(\theta)\boldsymbol{\hat{e_1}}+\sin(\theta)\boldsymbol{\hat{e_2}}\\
		\boldsymbol{\hat{\theta}} &= -\sin(\theta)\boldsymbol{\hat{e_1}} + \cos(\theta)\boldsymbol{\hat{e_2}}
	\end{align*}
	..där $\boldsymbol{\hat{e_1}}$ och $\boldsymbol{\hat{e_2}}$ är enhets-basvektorerna i standard (Kartesiska) koordinatsystem.\\\\
	
	Gradienten och nabla operatorn för en funktion $f : \mathbb{R}^2 \mapsto \mathbb{R}$ i termer av $x$ och $y$ definieras i termer av enhets-basvektorerna som:
	\begin{equation*}
		\nabla = \frac{\partial }{\partial x}\boldsymbol{\hat{e_1}}+\frac{\partial }{\partial y}\boldsymbol{\hat{e_2}} \implies \nabla f = \frac{\partial f}{\partial x}\boldsymbol{\hat{e_1}}+\frac{\partial f}{\partial y}\boldsymbol{\hat{e_2}}
	\end{equation*}
	En funktion $g : \mathbb{R}^2 \mapsto \mathbb{R}$ i termer av $r$ och $\theta$ beror av $x$ och $y$ eftersom:
	\begin{align*}
		r &= \sqrt{x^2+y^2}\\
		\theta &= arctan\bigl(\frac{y}{x}\bigr)
	\end{align*}
	Därmed partiell derivering med avseende på $x$ eller $y$ på en funktion som är angiven i polära koordinater (beror på $r$ och $\theta$ som i sin tur beror på $x$ och $y$ ) kommer vara definierad på sättet nedan:
	\begin{align*}
		\frac{\partial }{\partial x} = \frac{\partial }{\partial r} \frac{\partial r}{\partial x} + \frac{\partial }{\partial \theta} \frac{\partial \theta}{\partial x}\\
		\frac{\partial }{\partial y} = \frac{\partial }{\partial r} \frac{\partial r}{\partial y} + \frac{\partial }{\partial \theta} \frac{\partial \theta}{\partial y}
	\end{align*}
	Detta enligt kedjeregeln. $\frac{\partial r}{\partial x}, \frac{\partial \theta}{\partial x}, \frac{\partial r}{\partial y}, \frac{\partial \theta}{\partial y}$ kan då fås ut från definitionerna ovan, där $r(x,y) = \sqrt{x^2+y^2}$ och $\theta(x,y) = arctan\bigl(\frac{y}{x}\bigr)$:
	\begin{align*}
		\frac{\partial r}{\partial x} &= \frac{x}{\sqrt{x^2+y^2}} = \frac{r\cos(\theta)}{\sqrt{r^2}} = \cos(\theta) &\implies \frac{\partial }{\partial r} \frac{\partial r}{\partial x} &= \frac{\partial }{\partial r}\cos(\theta)\\
		\frac{\partial \theta}{\partial x} &= -\frac{y}{x^2} \cdot \frac{1}{\frac{y^2}{x^2}+1} = -\frac{y}{x^2+y^2} = -\frac{r\sin(\theta)}{r^2} = - \frac{\sin(\theta)}{r} &\implies \frac{\partial }{\partial \theta} \frac{\partial \theta}{\partial x} &= -\frac{\partial }{\partial \theta} \frac{\sin(\theta)}{r}\\
		\frac{\partial r}{\partial y} &= \frac{y}{\sqrt{x^2+y^2}} = \frac{r\sin(\theta)}{\sqrt{r^2}} = \sin(\theta) &\implies \frac{\partial }{\partial r} \frac{\partial r}{\partial y} &= \frac{\partial }{\partial r}\sin(\theta)\\
		\frac{\partial \theta}{\partial y} &= \frac{1}{x} \cdot \frac{1}{\frac{y^2}{x^2}+1} = \frac{x}{x^2+y^2} = \frac{r\cos(\theta)}{\sqrt{r^2}} = \frac{\cos(\theta)}{r} &\implies \frac{\partial }{\partial \theta} \frac{\partial \theta}{\partial y} &= \frac{\partial }{\partial \theta} \frac{\cos(\theta)}{r}
	\end{align*}
	Om vi följer definitionen för gradienten och nabla operatorn i standard koordinatsystemet och definitionerna av $\frac{\partial}{\partial x}$ och $\frac{\partial}{\partial y}$ för polära funktioner så får vi att:
	\begin{equation*}
		\nabla = \bigl(\frac{\partial }{\partial r}\cos(\theta)-\frac{\partial }{\partial \theta} \frac{\sin(\theta)}{r}\bigr)\boldsymbol{\hat{e_1}} + \bigl( \frac{\partial }{\partial r}\sin(\theta) + \frac{\partial }{\partial \theta} \frac{\cos(\theta)}{r} \bigr)\boldsymbol{\hat{e_2}}
	\end{equation*}
	Men vi vill använda de polära basvektorerna istället för $\boldsymbol{\hat{e_1}}$ och $\boldsymbol{\hat{e_2}}$. Vi kan bryta ut $\boldsymbol{\hat{e_1}}$ och $\boldsymbol{\hat{e_2}}$ i termer av $\boldsymbol{\hat{r}}$ och $\boldsymbol{\hat{\theta}}$:
	\begin{align*}
		\boldsymbol{\hat{e_1}} &= \cos(\theta)\boldsymbol{\hat{r}} - \sin(\theta)\boldsymbol{\hat{\theta}}\\
		\boldsymbol{\hat{e_2}} &= \sin(\theta)\boldsymbol{\hat{r}} + \cos(\theta)\boldsymbol{\hat{\theta}}
	\end{align*}
	Då blir nabla operatorn i polära koordinater och med polära basvektorer:
	\begin{align*}
	\nabla &= \bigl(\frac{\partial }{\partial r}\cos(\theta)-\frac{\partial }{\partial \theta} \frac{\sin(\theta)}{r}\bigr)\bigl(\cos(\theta)\boldsymbol{\hat{r}} - \sin(\theta)\bigr) + \bigl( \frac{\partial }{\partial r}\sin(\theta) + \frac{\partial }{\partial \theta} \frac{\cos(\theta)}{r} \bigr)\bigl(\sin(\theta)\boldsymbol{\hat{r}} + \cos(\theta)\bigr)\\
		&= \cdots\\
		&= \frac{\partial}{\partial r}\boldsymbol{\hat{r}} + \frac{1}{r}\frac{\partial}{\partial \theta}\boldsymbol{\hat{\theta}}
	\end{align*}
	Därmed gradienten för en funktion $h$ i polära koordinater blir:
	\begin{equation*}
		\nabla h = \frac{\partial h}{\partial r}\boldsymbol{\hat{r}} + \frac{1}{r}\frac{\partial h}{\partial \theta}\boldsymbol{\hat{\theta}}
	\end{equation*}
}


\end{document}
